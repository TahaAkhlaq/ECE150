\documentclass{article} % Use the article class for standard document layout
\usepackage{tikz}
\usetikzlibrary{matrix,calc}
\usepackage[margin=1cm]{geometry} % Adjust the margins as needed
\usepackage{xcolor} % Load without the dvipsnames option to avoid the clash

% Define dark green manually
\definecolor{darkgreen}{rgb}{0,0.5,0}

%isolated term
%#1 - Optional. Space between node and grouping line. Default=0
%#2 - node
%#3 - filling color
\newcommand{\implicantsol}[3][0]{
    \draw[rounded corners=3pt, fill=#3, opacity=0.3] ($(#2.north west)+(135:#1)$) rectangle ($(#2.south east)+(-45:#1)$);
    }


%internal group
%#1 - Optional. Space between node and grouping line. Default=0
%#2 - top left node
%#3 - bottom right node
%#4 - filling color
\newcommand{\implicant}[4][0]{
    \draw[rounded corners=3pt, fill=#4, opacity=0.3] ($(#2.north west)+(135:#1)$) rectangle ($(#3.south east)+(-45:#1)$);
    }

%group lateral borders
%#1 - Optional. Space between node and grouping line. Default=0
%#2 - top left node
%#3 - bottom right node
%#4 - filling color
\newcommand{\implicantcostats}[4][0]{
    \draw[rounded corners=3pt, fill=#4, opacity=0.3] ($(rf.east |- #2.north)+(90:#1)$)-| ($(#2.east)+(0:#1)$) |- ($(rf.east |- #3.south)+(-90:#1)$);
    \draw[rounded corners=3pt, fill=#4, opacity=0.3] ($(cf.west |- #2.north)+(90:#1)$) -| ($(#3.west)+(180:#1)$) |- ($(cf.west |- #3.south)+(-90:#1)$);
}

%group top-bottom borders
%#1 - Optional. Space between node and grouping line. Default=0
%#2 - top left node
%#3 - bottom right node
%#4 - filling color
\newcommand{\implicantdaltbaix}[4][0]{
    \draw[rounded corners=3pt, fill=#4, opacity=0.3] ($(cf.south -| #2.west)+(180:#1)$) |- ($(#2.south)+(-90:#1)$) -| ($(cf.south -| #3.east)+(0:#1)$);
    \draw[rounded corners=3pt, fill=#4, opacity=0.3] ($(rf.north -| #2.west)+(180:#1)$) |- ($(#3.north)+(90:#1)$) -| ($(rf.north -| #3.east)+(0:#1)$);
}

%group corners
%#1 - Optional. Space between node and grouping line. Default=0
%#2 - filling color
\newcommand{\implicantcantons}[2][0]{
    \draw[rounded corners=3pt, opacity=.3] ($(rf.east |- 0.south)+(-90:#1)$) -| ($(0.east |- cf.south)+(0:#1)$);
    \draw[rounded corners=3pt, opacity=.3] ($(rf.east |- 8.north)+(90:#1)$) -| ($(8.east |- rf.north)+(0:#1)$);
    \draw[rounded corners=3pt, opacity=.3] ($(cf.west |- 2.south)+(-90:#1)$) -| ($(2.west |- cf.south)+(180:#1)$);
    \draw[rounded corners=3pt, opacity=.3] ($(cf.west |- 10.north)+(90:#1)$) -| ($(10.west |- rf.north)+(180:#1)$);
    \fill[rounded corners=3pt, fill=#2, opacity=.3] ($(rf.east |- 0.south)+(-90:#1)$) -|  ($(0.east |- cf.south)+(0:#1)$) [sharp corners] ($(rf.east |- 0.south)+(-90:#1)$) |-  ($(0.east |- cf.south)+(0:#1)$) ;
    \fill[rounded corners=3pt, fill=#2, opacity=.3] ($(rf.east |- 8.north)+(90:#1)$) -| ($(8.east |- rf.north)+(0:#1)$) [sharp corners] ($(rf.east |- 8.north)+(90:#1)$) |- ($(8.east |- rf.north)+(0:#1)$) ;
    \fill[rounded corners=3pt, fill=#2, opacity=.3] ($(cf.west |- 2.south)+(-90:#1)$) -| ($(2.west |- cf.south)+(180:#1)$) [sharp corners]($(cf.west |- 2.south)+(-90:#1)$) |- ($(2.west |- cf.south)+(180:#1)$) ;
    \fill[rounded corners=3pt, fill=#2, opacity=.3] ($(cf.west |- 10.north)+(90:#1)$) -| ($(10.west |- rf.north)+(180:#1)$) [sharp corners] ($(cf.west |- 10.north)+(90:#1)$) |- ($(10.west |- rf.north)+(180:#1)$) ;
}

%Empty Karnaugh map 4x4
\newenvironment{Karnaugh}%
{
\begin{tikzpicture}[baseline=(current bounding box.north),scale=0.8]
\draw (0,0) grid (4,4);
\draw (0,4) -- node [pos=0.7,above right,anchor=south west] {CD} node [pos=0.7,below left,anchor=north east] {AB} ++(135:1);
%
\matrix (mapa) [matrix of nodes,
        column sep={0.8cm,between origins},
        row sep={0.8cm,between origins},
        every node/.style={minimum size=0.3mm},
        anchor=8.center,
        ampersand replacement=\&] at (0.5,0.5)
{
                       \& |(c00)| 00         \& |(c01)| 01         \& |(c11)| 11         \& |(c10)| 10         \& |(cf)| \phantom{00} \\
|(r00)| 00             \& |(0)|  \phantom{0} \& |(1)|  \phantom{0} \& |(3)|  \phantom{0} \& |(2)|  \phantom{0} \&                     \\
|(r01)| 01             \& |(4)|  \phantom{0} \& |(5)|  \phantom{0} \& |(7)|  \phantom{0} \& |(6)|  \phantom{0} \&                     \\
|(r11)| 11             \& |(12)| \phantom{0} \& |(13)| \phantom{0} \& |(15)| \phantom{0} \& |(14)| \phantom{0} \&                     \\
|(r10)| 10             \& |(8)|  \phantom{0} \& |(9)|  \phantom{0} \& |(11)| \phantom{0} \& |(10)| \phantom{0} \&                     \\
|(rf) | \phantom{00}   \&                    \&                    \&                    \&                    \&                     \\
};
}%
{
\end{tikzpicture}
}

%Empty Karnaugh map 2x4
\newenvironment{Karnaughvuit}%
{
\begin{tikzpicture}[baseline=(current bounding box.north),scale=0.8]
\draw (0,0) grid (4,2);
\draw (0,2) -- node [pos=0.7,above right,anchor=south west] {bc} node [pos=0.7,below left,anchor=north east] {a} ++(135:1);
%
\matrix (mapa) [matrix of nodes,
        column sep={0.8cm,between origins},
        row sep={0.8cm,between origins},
        every node/.style={minimum size=0.3mm},
        anchor=4.center,
        ampersand replacement=\&] at (0.5,0.5)
{
                      \& |(c00)| 00         \& |(c01)| 01         \& |(c11)| 11         \& |(c10)| 10         \& |(cf)| \phantom{00} \\
|(r00)| 0             \& |(0)|  \phantom{0} \& |(1)|  \phantom{0} \& |(3)|  \phantom{0} \& |(2)|  \phantom{0} \&                     \\
|(r01)| 1             \& |(4)|  \phantom{0} \& |(5)|  \phantom{0} \& |(7)|  \phantom{0} \& |(6)|  \phantom{0} \&                     \\
|(rf) | \phantom{00}  \&                    \&                    \&                    \&                    \&                     \\
};
}%
{
\end{tikzpicture}
}

%Empty Karnaugh map 2x2
\newenvironment{Karnaughquatre}%
{
\begin{tikzpicture}[baseline=(current bounding box.north),scale=0.8]
\draw (0,0) grid (2,2);
\draw (0,2) -- node [pos=0.7,above right,anchor=south west] {b} node [pos=0.7,below left,anchor=north east] {a} ++(135:1);
%
\matrix (mapa) [matrix of nodes,
        column sep={0.8cm,between origins},
        row sep={0.8cm,between origins},
        every node/.style={minimum size=0.3mm},
        anchor=2.center,
        ampersand replacement=\&] at (0.5,0.5)
{
          \& |(c00)| 0          \& |(c01)| 1  \\
|(r00)| 0 \& |(0)|  \phantom{0} \& |(1)|  \phantom{0} \\
|(r01)| 1 \& |(2)|  \phantom{0} \& |(3)|  \phantom{0} \\
};
}%
{
\end{tikzpicture}
}

%Defines 8 or 16 values (0,1,X)
\newcommand{\contingut}[1]{%
\foreach \x [count=\xi from 0]  in {#1}
     \path (\xi) node {\x};
}

%Places 1 in listed positions
\newcommand{\minterms}[1]{%
    \foreach \x in {#1}
        \path (\x) node {1};
}

%Places 0 in listed positions
\newcommand{\maxterms}[1]{%
    \foreach \x in {#1}
        \path (\x) node {0};
}

%Places X in listed positions
\newcommand{\indeterminats}[1]{%
    \foreach \x in {#1}
        \path (\x) node {X};
}



%------------------------------------------------------------------------------------------------------------------------------------------

\begin{document}

% Start of Karnaugh maps layout
\noindent % Ensures the line starts at the left margin
\begin{minipage}{0.5\textwidth} % Left column starts

    %Example
    \begin{center}
        \textbf{\underline{Example}:}
    \end{center}
    \vspace{-10mm}
        
    \begin{center}
    \begin{Karnaugh}
        \contingut{0,0,0,0,0,0,0,0,0,0,0,0,0,0,0,0}
        \implicant{0}{2}{red}
        \implicant{5}{15}{purple}
        \implicantdaltbaix[3pt]{3}{10}{blue}
        \implicantcantons[2pt]{orange}
       \implicantcostats{4}{14}{green}
    \end{Karnaugh}
    \end{center}
    %

%---------------------------------------KMAPS FOR 0s!-------------------------------------------

    %---------------------------------------R1!-------------------------------------------
    \begin{center}
        \textbf{\underline{\textcolor{red}{R}1}}\textbf{:}
    \end{center}
    \vspace{-10mm}

    %Kmap
    \begin{center}
    \begin{Karnaugh}
        \minterms{1,2,5,7,6,9} % Places 1 at these positions
        \implicant{2}{6}{blue}
        \implicant{7}{6}{red}
        \implicant{1}{5}{green}
        \implicantdaltbaix[3pt]{1}{9}{yellow}
    \end{Karnaugh}
    \end{center}
    \vspace{-10mm}

    % Boolean Expression
    \[
    \overline{B}\,\overline{C}D + \overline{A}\,\overline{C}\,D + \overline{A}BC + \overline{A}C\overline{D}
    \]
    \[
    \overline{C}D(\overline{A}+B)+\overline{A}C(B+\overline{D})
    \]
    %

    %------------------------------------------G1!-------------------------------------------
    \begin{center}
        \textbf{\underline{\textcolor{darkgreen}{G}1}}\textbf{:}
    \end{center}
    \vspace{-10mm}

    %Kmap
    \begin{center}
    \begin{Karnaugh}
        \minterms{0,4,3,8} % Places 1 at these positions
        \implicant{0}{4}{blue}
        \implicantsol[0]{3}{red} %TO CIRCLE 1 CELL!
        \implicantdaltbaix[3pt]{0}{8}{green} %TO CIRCLE OPPOSITE CELLS!
    \end{Karnaugh}
    \end{center}
    \vspace{-10mm}

    % Boolean Expression
    \[
    \overline{A}\,\overline{B}CD + \overline{A}\,\overline{C}\,\overline{D} + \overline{B}\,\overline{C}\,\overline{D}
    \]
    \[
    \overline{A}\,\overline{B}CD +\overline{C}\,\overline{D}(\overline{A}+\overline{B})
    \]
    \[
     \overline{A}\,\overline{B}CD +(\overline{C+D})(\overline{AB})
    \]
    %

    %------------------------------------------R2!-------------------------------------------
    \begin{center}
        \textbf{\underline{\textcolor{red}{R}2}}\textbf{:}
    \end{center}
    \vspace{-10mm}

    %Kmap
    \begin{center}
    \begin{Karnaugh}
        \minterms{1,3,7,8,9} % Places 1 at these positions
        \implicant{1}{3}{blue}
        \implicant{3}{7}{red}
        \implicant{8}{9}{green}

    \end{Karnaugh}
    \end{center}
    \vspace{-10mm}

    % Boolean Expression
    \[
    \overline{A}\,\overline{B}D + \overline{A}CD + A\overline{B}\,\overline{C}
    \]
    \[
    \overline{A}D(\overline{B} + C) + A\overline{B}\,\overline{C}
    \]
    %
    
\end{minipage}%
\hfill % This command adds a horizontal space between the two minipages that fills the available space
\begin{minipage}{0.5\textwidth} % Right column starts

    %------------------------------------------G2!-------------------------------------------
    \begin{center}
        \textbf{\underline{\textcolor{darkgreen}{G}2}}\textbf{:}
    \end{center}
    \vspace{-10mm}

    %Kmap
    \begin{center}
    \begin{Karnaugh}
        \minterms{4,5,6} % Places 1 at these positions
        \implicant{4}{5}{blue}
        \implicantcostats{4}{6}{green}
    \end{Karnaugh}
    \end{center}
    \vspace{-10mm}

    % Boolean Expression
    \[
    \overline{A}B\overline{C} + \overline{A}B\overline{D}
    \]
    \[
    \overline{A}B (\overline{C}+\overline{D})
    \]
    \[
    \overline{A}B (\overline{CD})
    \]
    %

    %------------------------------------------R3!-------------------------------------------
    \begin{center}
        \textbf{\underline{\textcolor{red}{R}3}}\textbf{:}
    \end{center}
    \vspace{-10mm}

    %Kmap
    \begin{center}
    \begin{Karnaugh}
        \minterms{1,6} % Places 1 at these positions
        \implicantsol[0]{1}{red} %TO CIRCLE 1 CELL!
        \implicantsol[0]{6}{blue} 
    \end{Karnaugh}
    \end{center}
    \vspace{-10mm}

    % Boolean Expression
    \[
    \overline{A}\,\overline{B}\,\overline{C}D + \overline{A}BC\,\overline{D}
    \]
    \[
    \overline{A}(\overline{B}\,\overline{C}D + BC\overline{D})
    \]
    %

    %------------------------------------------G3!-------------------------------------------
    \begin{center}
        \textbf{\underline{\textcolor{darkgreen}{G}3}}\textbf{:}
    \end{center}
    \vspace{-10mm}

    %Kmap
    \begin{center}
    \begin{Karnaugh}
        \minterms{3,4,5,7} % Places 1 at these positions
        \implicant{3}{7}{blue}
        \implicant{4}{5}{red} 
    \end{Karnaugh}
    \end{center}
    \vspace{-10mm}

    % Boolean Expression
    \[
    \overline{A}B\overline{C} + \overline{A}CD
    \]
    \[
    \overline{A} (B\overline{C} + CD)
    \]
    %
    
\end{minipage} % Right column ends

% Start of Karnaugh maps layout
\noindent % Ensures the line starts at the left margin
\begin{minipage}{0.5\textwidth} % Left column starts

    %------------------------------------------R4!-------------------------------------------
    \begin{center}
        \textbf{\underline{\textcolor{red}{R}4}}\textbf{:}
    \end{center}
    \vspace{-10mm}

    %Kmap
    \begin{center}
    \begin{Karnaugh}
        \minterms{1,3,5} % Places 1 at these positions
        \implicant{1}{3}{blue}
        \implicant{1}{5}{red}
    \end{Karnaugh}
    \end{center}
    \vspace{-10mm}

    % Boolean Expression
    \[
    \overline{A}\,\overline{C}D + \overline{A}\,\overline{B}D
    \]
    \[
    \overline{A}D (\overline{C} + \overline{B})
    \]
    \[
    \overline{A}D (\overline{CB})
    \]
    %

    %------------------------------------------G4!-------------------------------------------
    \begin{center}
        \textbf{\underline{\textcolor{darkgreen}{G}4}}\textbf{:}
    \end{center}
    \vspace{-10mm}

    %Kmap
    \begin{center}
    \begin{Karnaugh}
        \minterms{} % Places 1 at these positions
    \end{Karnaugh}
    \end{center}
    \vspace{-10mm}





%------------------------------------------KMAPS FOR Xs!------------------------------------------
    %--------------------------------------------R1!---------------------------------------------
    \begin{center}
        \textbf{\underline{\textcolor{red}{R}1}}\textbf{:}
    \end{center}
    \vspace{-10mm}

    %Kmap
    \begin{center}
    \begin{Karnaugh}
        \indeterminats{11, 10, 12, 13, 14, 15} % Places X at these positions
        \minterms{1,2,5,6,7,9} % Places 1 at these positions
        \implicant{1}{9}{blue}
        \implicant{7}{14}{red}
        \implicant{2}{10}{green}
    \end{Karnaugh}
    \end{center}
    \vspace{-10mm}

    % Boolean Expression
    \[
    C\overline{D} + CB + \overline{C}D
    \]
    \[
    CB + (C\oplus D)
    \]
    %

%--------------------------------------------G1!---------------------------------------------
    \begin{center}
        \textbf{\underline{\textcolor{darkgreen}{G}1}}\textbf{:}
    \end{center}
    \vspace{-10mm}

    %Kmap
    \begin{center}
    \begin{Karnaugh}
        \indeterminats{11, 10, 12, 13, 14, 15} % Places X at these positions
        \minterms{0,3,4,8} % Places 1 at these positions
        \implicant{0}{8}{blue}
        \implicantdaltbaix[3pt]{3}{11}{red} %TO CIRCLE OPPOSITE CELLS!
    \end{Karnaugh}
    \end{center}
    \vspace{-10mm}

    % Boolean Expression
    \[
    \overline{C}\,\overline{D} + \overline{B}CD
    \]
    %
    
\end{minipage} % Left column ends

\hfill % This command adds a horizontal space between the two minipages that fills the available space
\begin{minipage}{0.5\textwidth} % Right column starts

%--------------------------------------------R2!---------------------------------------------
    \begin{center}
        \textbf{\underline{\textcolor{red}{R}2}}\textbf{:}
    \end{center}
    \vspace{-10mm}

    %Kmap
    \begin{center}
    \begin{Karnaugh}
        \indeterminats{11, 10, 12, 13, 14, 15} % Places X at these positions
        \minterms{1, 3, 7, 8, 9} % Places 1 at these positions
        \implicant{12}{10}{blue}
        \implicant{3}{11}{red}
        \implicantdaltbaix[3pt]{1}{11}{green}
    \end{Karnaugh}
    \end{center}
    \vspace{-10mm}

    % Boolean Expression
    \[
    A + \overline{B}D + CD
    \]
    \[
    A + D (\overline{B} + C) 
    \]
    %

%--------------------------------------------G2!---------------------------------------------
    \begin{center}
        \textbf{\underline{\textcolor{darkgreen}{G}2}}\textbf{:}
    \end{center}
    \vspace{-10mm}

    %Kmap
    \begin{center}
    \begin{Karnaugh}
        \indeterminats{11, 10, 12, 13, 14, 15} % Places X at these positions
        \minterms{4, 5, 6} % Places 1 at these positions
        \implicant{4}{13}{blue}
        \implicant{6}{14}{red}
    \end{Karnaugh}
    \end{center}
    \vspace{-10mm}

    % Boolean Expression
    \[
     B\overline{C} + BC\overline{D}
    \]
    \[
     B(\overline{C}  + C\overline{D})
    \]
    \[
     B(\overline{C} + \overline{D}) 
    \]
    %

%--------------------------------------------R3!---------------------------------------------
    \begin{center}
        \textbf{\underline{\textcolor{red}{R}3}}\textbf{:}
    \end{center}
    \vspace{-10mm}

    %Kmap
    \begin{center}
    \begin{Karnaugh}
        \indeterminats{11, 10, 12, 13, 14, 15} % Places X at these positions
        \minterms{1, 6} % Places 1 at these positions
        \implicantsol[0]{1}{red} %TO CIRCLE 1 CELL!
        \implicantsol[0]{6}{blue} %TO CIRCLE 1 CELL!
    \end{Karnaugh}
    \end{center}
    \vspace{-10mm}

    % Boolean Expression
    \[
     \overline{A}\,\overline{B}\,\overline{C}D + \overline{A}BC\overline{D}
    \]
    \[
     \overline{A} (\overline{B}\,\overline{C}D + BC\overline{D}) 
    \]
    \[
      \overline{A}(B\oplus D)(C\oplus D)
    \]
    %

%--------------------------------------------G3!---------------------------------------------
    \begin{center}
        \textbf{\underline{\textcolor{darkgreen}{G}3}}\textbf{:}
    \end{center}
    \vspace{-10mm}

    %Kmap
    \begin{center}
    \begin{Karnaugh}
        \indeterminats{11, 10, 12, 13, 14, 15} % Places X at these positions
        \minterms{3, 4, 5, 7} % Places 1 at these positions
        \implicant{3}{11}{blue}
        \implicant{4}{13}{red}
    \end{Karnaugh}
    \end{center}
    \vspace{-10mm}

    % Boolean Expression
    \[
     B\overline{C} + CD
    \]
    %
    
\end{minipage} % Right Column Ends

% Start of Karnaugh maps layout
\noindent % Ensures the line starts at the left margin
\begin{minipage}{0.5\textwidth} % Left column starts

%--------------------------------------------R4!---------------------------------------------
    \begin{center}
        \textbf{\underline{\textcolor{red}{R}4}}\textbf{:}
    \end{center}
    \vspace{-10mm}

    %Kmap
    \begin{center}
    \begin{Karnaugh}
        \indeterminats{11, 10, 12, 13, 14, 15} % Places X at these positions
        \minterms{1, 3, 5} % Places 1 at these positions
        \implicant{1}{3}{blue}
        \implicant{1}{5}{red}
    \end{Karnaugh}
    \end{center}
    \vspace{-10mm}

    % Boolean Expression
    \[
     \overline{A}\,\overline{C}D + \overline{A}\,\overline{B}D
    \]
    \[
     \overline{A}D (\overline{C} + \overline{B}) 
    \]
    \[
      \overline{A}D (\overline{CB})
    \]
    %

\end{minipage} % Left Column Ends
    
\end{document}